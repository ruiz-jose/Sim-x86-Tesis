% This is the Reed College LaTeX thesis template. Most of the work
% for the document class was done by Sam Noble (SN), as well as this
% template. Later comments etc. by Ben Salzberg (BTS). Additional
% restructuring and APA support by Jess Youngberg (JY).
% Your comments and suggestions are more than welcome; please email
% them to cus@reed.edu
%
% See https://www.reed.edu/cis/help/LaTeX/index.html for help. There are a
% great bunch of help pages there, with notes on
% getting started, bibtex, etc. Go there and read it if you're not
% already familiar with LaTeX.
%
% Any line that starts with a percent symbol is a comment.
% They won't show up in the document, and are useful for notes
% to yourself and explaining commands.
% Commenting also removes a line from the document;
% very handy for troubleshooting problems. -BTS

% As far as I know, this follows the requirements laid out in
% the 2002-2003 Senior Handbook. Ask a librarian to check the
% document before binding. -SN

%%
%% Preamble
%%
% \documentclass{<something>} must begin each LaTeX document
\documentclass[12pt,twoside]{templates/unerthesis}
% Packages are extensions to the basic LaTeX functions. Whatever you
% want to typeset, there is probably a package out there for it.
% Chemistry (chemtex), screenplays, you name it.
% Check out CTAN to see: https://www.ctan.org/
%%
%\ifxetex
%  \usepackage{polyglossia}
%  \setmainlanguage{spanish}
  % Tabla en lugar de cuadro
%  \gappto\captionsspanish{\renewcommand{\tablename}{Tabla}
%          \renewcommand{\listtablename}{Índice de tablas}}
%\else
%  \usepackage[spanish,es-tabla]{babel}
%\fi
%\usepackage[spanish]{babel}
\usepackage{graphicx,latexsym}
\usepackage{amsmath}
\usepackage{amssymb,amsthm}
\usepackage{longtable,booktabs,setspace}
\usepackage{chemarr} %% Useful for one reaction arrow, useless if you're not a chem major
\usepackage[hyphens]{url}
% Added by CII
%\usepackage{hyperref}
\usepackage[colorlinks = true,
            linkcolor = blue,
            urlcolor  = blue,
            citecolor = blue,
            anchorcolor = blue]{hyperref}
\usepackage{lmodern}
\usepackage{float}
\floatplacement{figure}{H}
% End of CII addition
\usepackage{rotating}
\usepackage{placeins} % para fijar la posición de las tablas con \FloatBarrier


\usepackage[]{natbib}


% Next line commented out by CII
%\usepackage{biblatex}
%\usepackage{natbib}
% Comment out the natbib line above and uncomment the following two lines to use the new
% biblatex-chicago style, for Chicago A. Also make some changes at the end where the
% bibliography is included.
%\usepackage{biblatex-chicago}
%\bibliography{thesis}


% Added by CII (Thanks, Hadley!)
% Use ref for internal links
\renewcommand{\hyperref}[2][???]{\autoref{#1}}
\def\chapterautorefname{Chapter}
\def\sectionautorefname{Section}
\def\subsectionautorefname{Subsection}
% End of CII addition

% Added by CII
\usepackage{caption}
\captionsetup{width=5in}
% End of CII addition

% \usepackage{times} % other fonts are available like times, bookman, charter, palatino

% Syntax highlighting #22

% To pass between YAML and LaTeX the dollar signs are added by CII
\title{Herramienta de simulación para dar soporte a la enseñanza de arquitectura de computadoras}
\author{Ruiz Jose Maria}
% The month and year that you submit your FINAL draft TO THE LIBRARY (May or December)
\date{Fecha}
\division{}
\advisor{Director: Colombani Marcelo Alberto}
\institution{Universidad de Nacional de Entre Rios}
\degree{Maestría en Sistemas de Información}
%If you have two advisors for some reason, you can use the following
% Uncommented out by CII
% End of CII addition

%%% Remember to use the correct department!
\department{}
% if you're writing a thesis in an interdisciplinary major,
% uncomment the line below and change the text as appropriate.
% check the Senior Handbook if unsure.
%\thedivisionof{The Established Interdisciplinary Committee for}
% if you want the approval page to say "Approved for the Committee",
% uncomment the next line
%\approvedforthe{Committee}

% Added by CII
%%% Copied from knitr
%% maxwidth is the original width if it's less than linewidth
%% otherwise use linewidth (to make sure the graphics do not exceed the margin)
\makeatletter
\def\maxwidth{ %
  \ifdim\Gin@nat@width>\linewidth
    \linewidth
  \else
    \Gin@nat@width
  \fi
}
\makeatother

%Added by @MyKo101, code provided by @GerbrichFerdinands

\setlength\parindent{0pt}


% Added by CII

\providecommand{\tightlist}{%
  \setlength{\itemsep}{0pt}\setlength{\parskip}{0pt}}

\Acknowledgements{

}

\Dedication{

}

\Preface{

}

\Abstract{

}

% End of CII addition
%%
%% End Preamble
%%
%
\let\chaptername\relax
\begin{document}
\bibliographystyle{apalike}
% Everything below added by CII
  \maketitle

\frontmatter % this stuff will be roman-numbered
\pagestyle{empty} % this removes page numbers from the frontmatter



%  \hypersetup{linkcolor=black}
  \setcounter{tocdepth}{1}
  \setlength{\parskip}{0pt}
  \tableofcontents

\setlength\parskip{1em plus 0.1em minus 0.2em}

  \listoftables

  \listoffigures



\mainmatter % here the regular arabic numbering starts
\pagestyle{fancyplain} % turns page numbering back on

\hypertarget{resumen}{%
\chapter*{Resumen}\label{resumen}}
\addcontentsline{toc}{chapter}{Resumen}

Resumen aquí

\hypertarget{agradecimientos}{%
\chapter*{Agradecimientos}\label{agradecimientos}}
\addcontentsline{toc}{chapter}{Agradecimientos}

Agradecimientos aquí.

\hypertarget{capuxedtulo-1-introducciuxf3n}{%
\chapter{Capítulo 1 Introducción}\label{capuxedtulo-1-introducciuxf3n}}

En nuestra vida cotidiana, dependemos de diversos dispositivos, desde máquinas de trabajo hasta teléfonos y relojes inteligentes, que funcionan gracias a la arquitectura de las computadoras. Comprender los componentes y su funcionamiento nos permite diseñar, desarrollar e implementar aplicaciones más eficientes.

Los alumnos de la asignatura Arquitectura de Computadoras no solo deben conocer la estructura y el funcionamiento interno de la computadora, sino que, idealmente, deben tener una experiencia práctica activa con dicha arquitectura.

Para brindar esta experiencia, es fundamental contar con un laboratorio equipado con el hardware necesario y dedicar el tiempo suficiente para que los alumnos adquieran competencia en el uso de herramientas prácticas. Por esta razón, se han desarrollado numerosos simuladores que ayudan a los estudiantes a comprender el funcionamiento y la estructura de las computadoras, proporcionando valiosas experiencias de aprendizaje.
La presente tesis se enmarca dentro en la carrera Maestría en Sistemas de Información dictada en la Facultad de Ciencias de la Administración, se vincula directamente con el proyecto de investigación I/D novel PID-UNER 7065: ``Enseñanza/aprendizaje de asignatura Arquitectura de Computadoras con herramientas de simulación de sistemas de cómputos'', llevado a cabo en la Facultad de Ciencias de la Administración de la Universidad Nacional de Entre Ríos{[}1{]}.

La asignatura arquitectura de computadoras forma parte del plan de estudios de la carrera Licenciatura en Sistemas de la Facultad de Ciencias de la Administración, Universidad Nacional de Entre Ríos. El objetivo de la asignatura Arquitectura de Computadoras es conocer la estructura y funcionamiento de las computadoras, entender la ejecución lógica de un programa a nivel de instrucciones máquinas.

Como primer nivel de entendimiento los alumnos deben ser capaces de comprender que las computadoras son máquinas que toman datos del exterior, los procesan y obtienen los resultados almacenándolos en la memoria o enviándolos a un dispositivo de entrada y salida.

El procesamiento lo realiza por medio del procesador o CPU, y es allí donde radica la mayor complejidad para el alumno y donde se pueden generar dificultades para comprender su funcionamiento.

Se pueden explicar las partes del mismo, su funcionamiento, la interacción que realiza cada componente, también se puede enseñar un lenguaje de programación ensamblador y realizar las prácticas de programación, pero en general se ve que el alumno no logra llegar a una comprensión del funcionamiento.

Sin embargo, la utilización de simuladores permite afianzar los conocimientos de los temas vistos en las clases teóricas, a fin de evitar que los estudiantes desvíen su atención hacia el aprendizaje del simulador propiamente, se debe buscar que los simuladores tengan un manejo simple, intuitivo y visualmente atractivo, simplificando su aprendizaje de su uso.

La simulación es un término de uso diario en muchos contextos: medicina, militar, entretenimiento, educación, etc., debido a que permite ayudar a comprender cómo funciona un sistema, responder preguntas como ``qué pasaría si'', con el fin de brindar hipótesis sobre cómo o por qué ocurren ciertos fenómenos.

Para continuar, se define simulación como el proceso de imitar el funcionamiento de un sistema a medida que avanza en el tiempo. Entonces para llevar a cabo una simulación, es necesario desarrollar previamente un modelo conceptual que representa las características o comportamientos del sistema, mientras que la simulación representa la evolución del modelo a medida que avanza en el tiempo {[}2{]}, {[}3{]}, {[}4{]}.

Con los avances en el mundo digital, la simulación se ha convertido en una metodología de solución de problemas indispensable para ingenieros, docentes, diseñadores y gerentes. La complejidad intrínseca de los sistemas informáticos los hace difícil comprender y costosos de desarrollar sin utilizar simulación {[}3{]}.

Muchas veces en el ámbito educativo, resulta difícil transmitir fundamentos teóricos de la organización y arquitectura interna de las computadoras debido a la complejidad de los procesos involucrados. Si sólo incorporamos los medios de enseñanza tradicionales, como ser una pizarra, un libro de texto o diapositivas, los mismos tienen una capacidad limitada para representar estos fundamentos. En consecuencia, es imprescindible un alto nivel de abstracción por parte del alumno para desarrollar un modelo mental adecuado para capturar la organización y arquitectura interna de las computadoras {[}5{]}, {[}6{]}, {[}7{]}.

Es evidente la necesidad de utilizar nuevas tecnologías como recurso didáctico y como medio para la transferencia de conocimiento, ya que resultan de gran ayuda para que los alumnos relacionan conceptos abstractos con reales, permite situar al alumno en un contexto que imite algún aspecto de la realidad; en ese ambiente, el alumno podrá detectar problemáticas similares a las que podrían producirse en la realidad, logrando un mejor entendimiento por medio del trabajo exploratorio, inferencia, aprendizaje por descubrimiento y desarrollo de habilidades {[}8{]}, {[}9{]}.

Un simulador de arquitectura es una herramienta que imita el hardware de un sistema. El simulador se centra principalmente en la representación de los aspectos arquitectónicos y funciones del hardware simulado. El uso de herramientas de simulación permite realizar cambios, pruebas y ejecución de programas sin temor de dañar ningún componente o por falta de la computadora {[}10{]}.

Algunas herramientas ofrecen una representación en forma visual e interactiva de la organización y arquitectura interna de la computadora, facilitando así la comprensión de su funcionamiento. En este sentido, los simuladores juegan una pieza clave en el campo de la Arquitectura de Computadores, permitiendo conectar fundamentos teóricos con la experiencia práctica, simplificando abstracciones y facilitando la labor docente {[}11{]}, {[}12{]}, {[}13{]}, {[}14{]}.

El repertorio de instrucciones de la arquitectura x86 es la más utilizada en computadoras de escritorio y servidores del mundo. Inició con el procesador Intel 8086 en el año 1978 como arquitectura de 16 bits. Después evolucionó hasta una arquitectura de 32 bits cuando apareció el procesador Intel 80386 en el año 1985, denominada i386 o x86-32. AMD amplió esta arquitectura de 32 bits a una de 64 bits. Intel adoptó las extensiones de la arquitectura de AMD de 64 bits, también denominada AMD64 o Intel 64 {[}15{]}, {[}16{]}.

Un procesador x86-64 mantiene la compatibilidad con los modos x86 existentes de 16 y 32 bits, y permite ejecutar aplicaciones de 16 y 32 bits, como así también de 64 bits. Esta compatibilidad hacia atrás protege las principales inversiones en aplicaciones y sistemas operativos desarrollados para la arquitectura x86 {[}15{]}, {[}16{]}, {[}17{]}.

Por ello, la enseñanza de la arquitectura x86 es de gran relevancia en la asignatura Arquitecturas de Computadoras debido a los diferentes temas que aborda.

Los alumnos de la asignatura Arquitectura de Computadoras no solo deben conocer la estructura y el funcionamiento interno de la computadora, sino que, idealmente, deben tener una experiencia práctica activa con dicha arquitectura.

Para proporcionar esta experiencia es necesario un laboratorio con el hardware necesario y el tiempo para que los alumnos se vuelvan competentes en el uso de herramientas para trabajar con el hardware. Por este motivo, muchos simuladores han sido desarrollados, ayudando al alumno a comprender el funcionamiento y la estructura del computador proporcionando valiosas experiencias de aprendizaje {[}18{]}.

\hypertarget{justificaciuxf3n}{%
\section{Justificación}\label{justificaciuxf3n}}

Aunque ya existen simuladores de la arquitectura x86 que apoyan la enseñanza en los cursos de Arquitectura de Computadoras {[}10{]}, {[}11{]}, estos suelen presentar una gran cantidad de contenidos preestablecidos. Aunque estos contenidos son relevantes, ofrecer toda la especificación de la arquitectura x86 desde el principio puede ser abrumador para los estudiantes y dificultar su comprensión.

Sin embargo, desde esta tesis se propone un enfoque diferente, el objetivo principal es construir un herramienta de simulación de la arquitectura x86 para dar apoyo a la enseñanza de arquitectura de computadoras, más concretamente de los contenidos específicos que se enseñan en la currícula de arquitectura de computadoras basados en la arquitectura x86. Partiendo de una visión global de la estructura y funcionamiento de la computadora (CPU, memoria, módulo de E/S y buses), mostrando los micropasos necesarios para la realización del ciclo básico de una instrucción, ofreciendo un repertorio reducido de instrucciones que se habiliten las instrucciones a medida que se dictan en la asignatura, permitiendo la generación y ejecución de programas escritos en lenguaje ensamblador, ya sea paso a paso por instrucción o completa, gestión básica de interrupciones permitiendo la interacción con el teclado y la pantalla, comunicación con los módulos de entrada y salida e interacciones con los periféricos, y por último, medidas de rendimiento sobre la ejecución de un programa.

Con el objeto de ofrecer al alumno un simulador bien diseñado, robusto, modular y por tanto flexible y sencillo de modificar o ampliar, se explorará la utilización de técnicas formales de modelización y simulación como las redes de Petri o DEVS (Discrete Event System Specification). Estas técnicas permiten separar conceptualmente las capas de modelización y simulación y ofrecen por ello una separación ortogonal de ambas, facilitando la comprensión y modificación del software. Además, permiten el escalado transparente de las simulaciones, pudiéndose ejecutar en entornos de cómputo paralelo o distribuido sin necesidad de modificar el modelo en sí, lo que aporta grandes ventajas de escalado {[}19{]}, {[}20{]}, {[}21{]}.

\hypertarget{objetivos}{%
\section{Objetivos}\label{objetivos}}

\hypertarget{objetivo-general}{%
\subsection{Objetivo General}\label{objetivo-general}}

El objetivo principal de esta tesis es desarrollar una herramienta de simulación de la arquitectura x86 para apoyar la enseñanza de arquitectura de computadoras, enfocándose específicamente en los contenidos de la currícula de Arquitectura de Computadoras basados en la arquitectura x86

\hypertarget{objetivos-especuxedficos}{%
\subsection{Objetivos Específicos}\label{objetivos-especuxedficos}}

Para ello se debe cumplir los siguientes objetivos específicos:
-Estudiar y evaluar diferentes herramientas actuales de simulación destinadas a dar apoyo a la enseñanza de la arquitectura x86.
-Construir una herramienta de apoyo para impartir los contenidos de la asignatura Arquitectura de Computadoras, para ello debe cumplir:
- Ofrecer una visión global de la estructura y funcionamiento de la computadora.
- Permitir la generación y ejecución de programas escritos en lenguaje ensamblador, ya sea paso a paso por instrucción o completa.
- Ofrecer un repertorio de instrucciones x86 reducido donde se habiliten las instrucciones a medida que se desarrolle el contenido en la asignatura.
- Simular de manera visual e interactiva los micros pasos que conlleva el ciclo básico de una instrucción.
- Permitir la gestión básica de interrupciones permitiendo la interacción con el teclado y la pantalla.
- Permitir la comunicación con los módulos de entrada y salida e interacciones con los periféricos.
- Ofrecer medidas de rendimiento sobre la ejecución de un programa.

\hypertarget{metodologuxeda-de-desarrollo}{%
\section{Metodología de Desarrollo}\label{metodologuxeda-de-desarrollo}}

Teniendo en cuenta los objetivos propuestos en la sección anterior, se pretende alcanzar los mismos a través de los pasos que se describen en esta sección.

\hypertarget{etapas-de-la-investigaciuxf3n}{%
\subsection{Etapas de la Investigación}\label{etapas-de-la-investigaciuxf3n}}

\begin{enumerate}
\def\labelenumi{\alph{enumi}.}
\tightlist
\item
  Análisis bibliográfico.
  Se realizó una revisión continua de las publicaciones científicas y tecnológicas, libros e informes técnicos relacionados con el objeto de estudio.
\item
  Recopilación de simuladores.
  Se realizó un relevamiento del estado actual y las actualizaciones de los simuladores aplicados a la enseñanza de arquitectura de computadoras.
\item
  Estudio de los simuladores.
  En base a la documentación relevada de los simuladores se estudió en profundidad al menos 5 simuladores y se elaboró una comparativa de los simuladores seleccionados en cuanto a los contenidos que se imparten en la asignatura.
\item
  Construir el simulador.
  A través de métodos y técnicas de ingeniería de software se construyó un simulador de la arquitectura x86 donde abarque los aspectos más relevantes de la asignatura Arquitectura de Computadoras, permitiendo desarrollar los contenidos en una plataforma unificada, evitando así la pérdida de tiempo y dificultad que supone para el alumno habituarse a diferentes entornos. Se utilizarán para ello técnicas formales de modelización y simulación, que facilitan un desarrollo modular y enfocan el esfuerzo en la definición del modelo de la arquitectura x86 más que en el protocolo de simulación, permitiendo además el escalado a entornos de ejecución paralelos o distribuidos sin necesidad de modificar el modelo de la arquitectura simulada.
\end{enumerate}

\hypertarget{organizaciuxf3n-del-documento}{%
\section{Organización del Documento}\label{organizaciuxf3n-del-documento}}

El resto de este documento se organiza de la siguiente manera: el capítulo 2 define formalmente las características y el set de instrucciones de la arquitectura. Luego, el capítulo 3 repasa y motiva el interesante rol que la simulación desde un punto de vista didáctico puede desempeñar para el dictado de la asignatura donde se abordan estos tópicos. El capítulo 4 comparativo de los simuladores estudiados según criterios preestablecidos. Finalmente, el capítulo 5 adaptación de un simulador como soporte para el uso de la enseñanza y aprendizaje de arquitectura de computadora.

\hypertarget{antecedentes-conceptuales-y-empuxedricos}{%
\chapter{Antecedentes conceptuales y empíricos}\label{antecedentes-conceptuales-y-empuxedricos}}

\hypertarget{concepto-principal---explanandum---objeto-de-estudio}{%
\section{(concepto principal - explanandum - objeto de estudio)}\label{concepto-principal---explanandum---objeto-de-estudio}}

\ldots{}

\hypertarget{factores-asociados-a-objeto-de-estudio}{%
\section{Factores asociados a objeto de estudio}\label{factores-asociados-a-objeto-de-estudio}}

\hypertarget{factor-1}{%
\subsection{Factor 1}\label{factor-1}}

Al final de esta sección, enunciar la hipótesis correspondiente

\hypertarget{factor-n}{%
\subsection{Factor N}\label{factor-n}}

Al final de esta sección, enunciar la hipótesis correspondiente

\hypertarget{metodologuxeda}{%
\chapter{Metodología}\label{metodologuxeda}}

\hypertarget{datos}{%
\section{Datos}\label{datos}}

\hypertarget{variables}{%
\section{Variables}\label{variables}}

\hypertarget{muxe9todos}{%
\section{Métodos}\label{muxe9todos}}

\hypertarget{anuxe1lisis}{%
\chapter{Análisis}\label{anuxe1lisis}}

\hypertarget{anuxe1lisis-descriptivo}{%
\section{Análisis descriptivo}\label{anuxe1lisis-descriptivo}}

\hypertarget{modelos}{%
\section{Modelos}\label{modelos}}

\hypertarget{conclusiones}{%
\chapter{Conclusiones}\label{conclusiones}}

\hypertarget{bibliografuxeda}{%
\chapter*{Bibliografía}\label{bibliografuxeda}}
\addcontentsline{toc}{chapter}{Bibliografía}

% %%%%%%%%%%%%%%%%%%%%%%%%%%%%%%%%%%%%%%%%%%%%%%%%%
% %%% Bibliography                              %%%
% %%%%%%%%%%%%%%%%%%%%%%%%%%%%%%%%%%%%%%%%%%%%%%%%%
% \addtocontents{toc}{\vspace{.5\baselineskip}}
% \cleardoublepage
% \phantomsection
% \addcontentsline{toc}{chapter}{\protect\numberline{}{Bibliography}}
\bibliography{tesis}

%% All books from our library (SfS) are already in a BiBTeX file
%% (Assbib). You can use Assbib combined with your personal BiBTeX file:
%% \bibliography{Myreferences,Assbib}. Of course, this will only work on
%% the computers at SfS, unless you copy the Assbib file
%%  --> /u/sfs/bib/Assbib.bib



\end{document}
